\section{无类型$\lambda$演算}

本节由txyyss和rainoftime提供。我们力求严格、形式化地介绍无类型$\lambda$演算,在这部分把自由变量、$\alpha$变换、$\beta$ 规约等基础概念说明清楚,也方便后续内容更集中在要点上。读者可选择性阅读。






\begin{defn}{\bfseries{($\lambda$ 项)}}\label{def:lambdaterm}

  假设我们有一个无穷的字符串集合,里面的元素被称为\emph{变量}(和程序
    语言中变量概念不同,这里就是指字符串本身)。那么 \emph{$\lambda$ 项}定义如下:
  \begin{tightenum}
  \item 所有的变量都是 $\lambda$ 项(名为\emph{原子});
  \item 若 $M$ 和 $N$ 是 $\lambda$ 项,那么 $(M\,N)$ 也是 $\lambda$ 项
    (名为\emph{应用})
  \item 若 $M$ 是 $\lambda$ 项而 $\phi$ 是一个变量,那么 $(\lambda\phi.M)$
    也是 $\lambda$ 项(名为\emph{抽象})。
  \end{tightenum}
\end{defn}



\begin{exmp}{\bfseries{(一些 $\lambda$ 项)}}
  下面这些都是 $\lambda$ 项:
  \begin{center}
    \begin{tabular*}{.8\textwidth}{@{\extracolsep{\fill} }ccc}
      $(\lambda x.(x\,y))$ & $(x(\lambda x.(\lambda x.x)))$ & $((((a\,b)\,c)\,d)\,e)$\\
      $(((\lambda x.(\lambda y.(y\,x)))\,a)\,b)$ & $((\lambda y.y)\,(\lambda x.(x\,y)))$ & $(\lambda x.(y\,z))$
    \end{tabular*}
  \end{center}
\end{exmp}

$\lambda$ 项是一种形式语言,换句话说,就是一类特殊形式的字符串罢,没
有任何内在的意义,只是个``形式''。通常情况下,当讨论一个形式语言的时候,
我们需要用另一种元语言来指称形式语言里的元素。就如讨论自然数时,我们经
常说``对任意自然数 $n$'',这里的 $n$ 本身并不是自然数,用来指称自然数
罢了。我们也需要一些``$n$''来表示 $\lambda$ 项中的元素。因此,我们作如
下符号约定:

\begin{notation}
  本节中我们用大写英文字母表示任意 $\lambda$ 项,用除 $\lambda$ 以外的
  小写希腊字母如 $\phi$,$\psi$ 等表示任意 $\lambda$ 项中的变量。

  对于括号,则有如下的省略规定:
  \begin{tightenum}
  \item $\lambda$ 项中最外层的括号可以省略,如 $(\lambda x.x)$ 可以省
    略表示为 $\lambda x.x$;
  \item 左结合的应用型的 $\lambda$ 项,如 $(((M\,N)\,P)\,Q)$,括号可
    以省略,表示为 $M\,N\,P\,Q$;
  \item 抽象型的 $\lambda$ 项 $(\lambda \phi.M)$ 中,$M$ 最外层的括号可以省略,
    如 $\lambda x.(y\,z)$ 可以省略为 $\lambda x.y\,z$。
  \end{tightenum}
  也就是说,我们把省略形式视同定义 \ref{def:lambdaterm} 中的 $\lambda$ 项。
\end{notation}

\begin{exmp}{\bfseries{(省略表示)}}
  下面给出了一些省略表示的 $\lambda$ 项。
  \begin{center}
    \begin{tabular*}{.7\textwidth}{@{\extracolsep{\fill} }ll}
      省略表示 & 完整的 $\lambda$ 项\\
      $\lambda x.\lambda y.y\,x\,a\,b$ & $(\lambda x.(\lambda y.(((y\,x)\,a)\,b)))$\\
      $(\lambda x.\lambda y.y\,x)\,a\,b$ & $(((\lambda x.(\lambda y.(y\,x)))\,a)\,b)$\\
      $\lambda g.(\lambda x.g\,(x\,x))\,\lambda x.g\,(x\,x)$ & $(\lambda g.((\lambda x.(g\,(x\,x)))\,(\lambda x.(g\,(x\,x)))))$\\
      $\lambda x.\lambda y.a\,b\,\lambda z.z$ & $(\lambda x.(\lambda y.((a\,b)\,(\lambda z.z))))$\\
    \end{tabular*}
  \end{center}
\end{exmp}



\subsection{$\alpha$替换}

这一小节将形式化的定义什么是
$\lambda$ 项的替换操作。直观的来讲,就是把 $\lambda$ 项中不被
$\lambda\phi$ 约束的变量 $\phi$ 替换成另一个 $\lambda$ 项罢了。但这么
一个操作的精确定义却不是直接和易于理解的。

\begin{defn}{\bfseries{(语法全等)}}

  我们用恒等号``$\equiv$''表示两个 $\lambda$ 项完全相同。换句话说
  $$
  M\equiv N
  $$表示 $M$ 和 $N$ 有完全相同的结构,且对应位置上的变量也完全相同。这意
  味着若 $M\,N\equiv P\,Q$ 则 $M\equiv P$ 且 $N\equiv Q$,若
  $\lambda\phi.M\equiv\lambda\psi.P$ 则 $\phi\equiv\psi$ 且 $M\equiv P$。
\end{defn}

\begin{defn}{\bfseries{(自由变量)}}

  对一个 $\lambda$ 项 $P$,我们可以定义 $P$ 中\emph{自由变量}的集合
  $\text{FV}(P)$ 如下:
  \begin{tightenum}
  \item $\text{FV}(\phi) = \{\phi\}$
  \item $\text{FV}(\lambda\phi.M) = \text{FV}(M)\,\backslash\,\{\phi\}$\label{enum:fvforabs}
  \item $\text{FV}(M\,N) = \text{FV}(M)\cup\text{FV}(N)$
  \end{tightenum}
\end{defn}

从第 \ref{enum:fvforabs} 可以看出抽象 $\lambda\phi.M$ 中的变量 $\phi$
是要从 $M$ 中被排除出自由变量这个集合的。若 $M$ 中有 $\phi$,我们可以
说它是被\emph{约束}的。据此可以进一步定义\emph{约束变量}集合。值得注意
的是,对同一个 $\lambda$ 项来说,这两个集合的交集未必为空。

\begin{exmp}{\bfseries{(自由变量)}}
  \begin{center}
    \begin{tabular*}{.7\textwidth}{@{\extracolsep{\fill} }ll}
      $\lambda$ 项 $P$ & 自由变量集合 $\text{FV}(P)$\\
      $\lambda x.\lambda y.x\,y\,a\,b$ & $\{a,b\}$\\
      $a\,b\,c\,d$ & $\{a,b,c,d\}$\\
      $x\,y\,\lambda y.\lambda x.x$ & $\{x,y\}$
    \end{tabular*}
  \end{center}
\end{exmp}

上面最后一个例子里,最左边的 $x,y$ 是自由变量,而最右侧的 $x$ 则是约束
变量。若对 $\lambda$ 项 $P$ 有 $\text{FV}(P)=\emptyset$,则称 $P$ 是
\emph{封闭}的,这样的 $P$ 又称为\emph{组合子}。

\begin{defn}{\bfseries{(出现)}}

  对于 $\lambda$ 项 $P$ 和 $Q$,可以定义一个二元关系\emph{出现}。我们
  说 $P$ 出现在 $Q$ 中,是这样定义的:
  \begin{tightenum}
  \item $P$ 出现在 $P$ 中;
  \item 若 $P$ 出现在 $M$ 中或 $N$ 中,则 $P$ 出现在 $(M\,N)$ 中;
  \item 若 $P$ 出现在 $M$ 中或 $P\equiv\phi$,则 $P$ 出现在 $(\lambda\phi.M)$ 中。
  \end{tightenum}
\end{defn}

有了上面这些定义,我们终于可以定义什么叫 $\lambda$ 项的替换操作了:
\begin{defn}{\bfseries{(替换)}}\label{def:sub}

  对任意 $M,N,\phi$,定义 $[N/\phi]\,M$ 是把 $M$ 中出现的自由变量
  $\phi$ 替换成 $N$ 后得到的结果,这个替换有可能改变部分约束变量名称以
  避免冲突。具体精确定义是一个对 $M$ 的归纳定义:
  \begin{tightenum}
  \item $[N/\phi]\,\phi\equiv N$
  \item $[N/\phi]\,\alpha\equiv\alpha$\hfill 对所有满足
    $\alpha\not\equiv\phi$ 的原子 $\alpha$
  \item $[N/\phi]\,(P\,Q)\equiv([N/\phi]\,P\;[N/\phi]\,Q)$\label{enum:sub:app}
  \item $[N/\phi]\,(\lambda\phi.P)\equiv\lambda\phi.P$\label{enum:sub:samebind}
  \item $[N/\phi]\,(\lambda\psi.P)\equiv\lambda\psi.P$\hfill 若
    $\phi\not\in\text{FV}(P)$\label{enum:sub:bind}
  \item
    $[N/\phi]\,(\lambda\psi.P)\equiv\lambda\psi.[N/\phi]\,P$\hfill
    若 $\phi\in\text{FV}(P)$ 且 $\psi\not\in\text{FV}(N)$\label{enum:sub:1free}
  \item
    $[N/\phi]\,(\lambda\psi.P)\equiv\lambda\eta.[N/\phi][\eta/\psi]\,P$\hfill
    若 $\phi\in\text{FV}(P)$ 且 $\psi\in\text{FV}(N)$\label{enum:sub:2free}
  \end{tightenum}
  对其中从 \ref{enum:sub:bind} 到 \ref{enum:sub:2free} 的各条来说,
  $\phi\not\equiv\psi$;而对 \ref{enum:sub:2free},$\eta$ 是满足
  $\eta\not\in\text{FV}(N\,P)$ 的任意变量。
\end{defn}

下面解释一下这个看上去很长很恐怖的定义,其实只要带着``替换 $\lambda$
项中的自由变量 $\phi$ 为 $N$''这个直观去看,就不难理解。前两条无非是说,
要是一个原子刚好是要被替换的变量,那就替换,不然就不动。第
\ref{enum:sub:app} 条也好说,遇到应用了,那就替换各子项。第
\ref{enum:sub:samebind} 条,$P$ 中的 $\phi$ 是被约束的,不能替换所以结
果不变。同样的情况发生在第 \ref{enum:sub:bind} 条,$P$ 中没有自由变量
$\phi$,结果也不变。顺着这个思路往下想,$P$ 中要是有自由变量 $\phi$,
是不是就可以替换了呢?是的,这就是第 \ref{enum:sub:1free} 条的内容了。
不过这条还有个附加条件,$\psi\not\in\text{FV}(N)$,为什么呢?因为 $P$
是被 $\lambda\psi$ 约束着的,把 $N$ 替换进去的时候,要是
$\text{FV}(N)$ 里面有 $\psi$,那替换进去的 $N$ 中的 $\psi$ 就从自由变
量变成约束变量了。这多少有点让人不放心,所以我们先把这个条件加上,这样
替换不会把自由变量变成约束变量,这就是第 \ref{enum:sub:1free} 了。但要
是不如意事长八九,$\text{N}$ 中偏偏有 $\psi$ 怎么办呢?为了避免这个冲
突,干脆把原先约束的变量先换成绝对不会起冲突的吧,找一个
$\eta\not\in\text{FV}(N\,P)$ 换掉 $\psi$,然后再按第
\ref{enum:sub:1free} 条来办,这就是第 \ref{enum:sub:2free} 的内容了。
特别注意这一条中,不仅把 $P$ 中的 $\psi$ 换成了 $\eta$,连最前面的
$\lambda\psi$ 都被换成了 $\lambda\eta$。

上面的解释只是为了让读者理解一下定义 \ref{def:sub} 的合理性,其实还是
有一些没有解释清楚的疑问,比如为什么从自由变量变成约束变量就不好了?第
\ref{enum:sub:2free} 条中 $\eta$ 有无数种选择,这种不确定性会不会有什
么问题?这些问题的答案是:没什么为什么,定义如此。定义不是定理,它说了
算,就算有它内蕴的合理性,也并不需要在这个时候证明。

定义 \ref{def:sub} 的第 \ref{enum:sub:2free} 条里,替换的不确定性暗示
了可以任意替换约束变量而不改变``意义''。这个可以类比的理解,例如定积分
$$
\int_a^bf(x)\text{d}x
$$里的 $x$ 是所谓的哑变量,换成别的如 $f(t)\text{d}t$ 不会改变积分式的
值。又比如正弦函数是写成 $\sin x$ 还是 $\sin t$ 都不影响它的意义。不过
这只是我为了能直观理解而加上的一种解释。实际上 $\lambda$ 项是形式系统,
只是形式,一堆字符串罢了,哪里有什么``意义''。

可以说为了补救任意替换约束变量给人带来的不安,我们有了下面的定义:

\begin{defn}{\bfseries{($\alpha$ 变换和 $\alpha$ 等价)}}
  设 $\lambda\phi.M$ 出现在一个 $\lambda$ 项 $P$ 中,且设
  $\psi\not\in\text{FV}(M)$,那么把 $\lambda\phi.M$ 替换成
  $$
  \lambda\psi.[\psi/\phi]\,M
  $$的操作被称为 $P$ 的 \emph{$\alpha$ 变换}。当且仅当若 $P$ 经过有限
  步(包括零步)$\alpha$ 变换后,得到新的 $\lambda$ 项 $Q$,则我们可以
  称 $P$ 与 $Q$ 是 $\alpha$ 等价的,又写作
             $$ P\quad\equiv_\alpha\quad Q
             $$
\end{defn}

\begin{exmp}
  $$
  \lambda x.\lambda y.x(x\,y) \equiv_\alpha\lambda x.\lambda v.x(x\,v) \equiv_\alpha\lambda u.\lambda v.u(u\,v)
  $$
\end{exmp}

容易证明,$\equiv_\alpha$ 满足自反性、对称性和传递性,所以确实是等价关
系。有了 $\alpha$ 等价,定义 \ref{def:sub} 就显得更加合理了,第
\ref{enum:sub:2free} 带来的疑问也得到解答了:不是担心任意替换不妥当么?
那么我们可以把 $\alpha$ 等价的 $\lambda$ 项都看成相同的,这不就``补救''了
由于不确定替换带来的问题嘛。事实上,有更好的结果:

\begin{thm}
  若 $M\,\equiv_\alpha\,M'$ 且 $N\,\equiv_\alpha\, N'$,则
  $[N/x]\,M\,\equiv_\alpha\,[N'/x]\,M'$
\end{thm}

强调一下,这个小节里定义的 $[N/\phi]\,M$ 并不是 $\lambda$ 项这个形式语
言里的东西,它只是一种记号,用来指称替换后的 $\lambda$ 项而已,切记切
记。而我各种解释说明中的意义和类比都只是为了方便直观的理解,并不是说
$\lambda$ 项有这些``意义''。它始终只是``名'',和``实''不相关。作来体现。

\subsection{$\beta$规约}

现在我们有了 $\lambda$ 项,有了替换的规则,那么什么时候需要进行替换呢?
这就是本小节要讨论的话题:规约。有了规约,整个关于 $\lambda$ 项的形式
系统才算完整,才可以说是 $\lambda$ 演算。

先介绍 $\beta$ 规约,$\beta$ 规约可以这么直观的理解:我们可以把
$(\lambda\phi.M)$ 看成是参数为 $\phi$,函数体为 $M$ 的一个函数;把
$(M\,N)$ 看成是函数 $M$ 作用到实际参数 $N$ 上\footnote{还是再强调一下,
  这只是直观的理解,并不是说两种 $\lambda$ 项一个是函数,一个是函数应
  用。$\lambda$ 项不是这些意义,只是字符串。}。平时我们要是定义了函数
$f(x)=x+5$,那么函数应用 $f(6)$ 就是把 $x+5$ 中的 $x$ 替换成 6,得到
$f(6)=6+5=11$。替换是函数应用的实质啊。

\begin{defn}{\bfseries{($\beta$ 规约)}}\label{def:betareduce}
  形如
  $$
  (\lambda\phi.M)\,N
  $$
  的 $\lambda$ 项被称为 \emph{$\beta$ 可约式},对应的项
  $$
  [N/\phi]\,M
  $$则称为 \emph{$\beta$ 缩减项}。当 $P$ 中含有 $(\lambda\phi.M)\,N$时,
  我们可以把 $P$ 中的 $(\lambda\phi.M)\,N$ 整体替换成$[N/\phi]\,M$,用
  $R$ 指称替换后的得到的项,那么我们说 $P$ 被 \emph{$\beta$ 缩减}为
  $R$,写做:
  $$
  P\;\triangleright_{1\beta}\;R
  $$当 $P$ 经过有限步(包括零步)的 $\beta$ 缩减后得到 $Q$,则称 $P$
  被 \emph{$\beta$ 规约}到 $Q$,写做:
  $$
  P\;\triangleright_\beta\;Q
  $$
\end{defn}

\newcommand{\betac}{\triangleright_{1\beta}}
\newcommand{\betar}{\triangleright_\beta}
\newcommand{\betae}{=_\beta}

\begin{exmp}{\bfseries{($\beta$ 缩减)}}\label{exmp:beta}
  \begin{alignat*}{2}
    (\lambda x.x\,(x\,y))\,m \quad &\betac\quad  m(m\,y) \\
    (\lambda x.y)\,n \quad &\betac\quad y \\
    (\lambda x.(\lambda y.y\,x)z)\,v\quad &\betac\quad [v/x]\,((\lambda y.y\,x)z) & &\equiv\quad (\lambda y.y\,v)z\\
    &\betac\quad [z/y]\,(y\,v)&\quad &\equiv\quad z\,v \\
    (\lambda x.x\,x)\,(\lambda x.x\,x)\quad &\betac\quad [(\lambda x.x\,x)/x]\,(x\,x) & &\equiv\quad (\lambda x.x\,x)\,(\lambda x.x\,x)\\
    &\betac\quad [(\lambda x.x\,x)/x]\,(x\,x) & &\equiv\quad (\lambda x.x\,x)\,(\lambda x.x\,x)\\
    &\cdots\quad \text{etc.}\\
    (\lambda x.x\,x\,y)\,(\lambda x.x\,x\,y)\quad
    &\betac\quad (\lambda x.x\,x\,y)\,(\lambda x.x\,x\,y)\,y\\
    &\betac\quad (\lambda x.x\,x\,y)\,(\lambda x.x\,x\,y)\,y\,y \\
    &\cdots\quad \text{etc.} \\
  \end{alignat*}
\end{exmp}

从上面的例子可以看出,$\beta$ 缩减虽然名为缩减,但实际情况是复杂的。像
最后这两个例子,一个永远缩减为自身,一个甚至更糟糕,越来越长,非但不是缩
减反而是增长了。

\begin{defn}{\bfseries{($\beta$ 范式)}}

  若一个 $\lambda$ 项 $Q$ 不含有 $\beta$ 可约式,则称 $Q$ 为
  \emph{$\beta$ 范式}。若有 $P$ 可被 $\beta$ 规约到 $Q$,则称 $Q$ 是
  $P$ 的 $\beta$ 范式。
\end{defn}

从示例 \ref{exmp:beta} 可以看出,并不是所有的 $\lambda$ 项都能
$\beta$ 规约到 $\beta$ 范式。这其实还不是最让人担心的,有一个之前读者
也许没注意到的问题:定义 \ref{def:betareduce} 里关于 $\beta$ 规约的定
义非常``狡猾'',或者叫非常含糊,它只要求存在一个有限长的 $\beta$ 缩减
链就行,当 $\lambda$ 项中含有不止一个 $\beta$ 可约式的时候,可以有不同
顺序的缩减方式。

\begin{exmp}\label{exmp:difforder}
  让我们重新看一下示例 \ref{exmp:beta} 里的 $(\lambda x.(\lambda
  y.y\,x)z)\,v$,最外层的 $(\lambda\dots)\,v$ 是一个 $\beta$ 可约式,
  里层的 $(\lambda y.y\,x)\,z$ 也是一个 $\beta$ 可约式。那么就有两种缩
  减方式:
  \begin{alignat*}{2}
    (\lambda x.(\lambda y.y\,x)z)\,v\quad &\betac \quad (\lambda y.y\,v)\,z \quad& &\text{缩减~}(\lambda x.(\lambda y.y\,x)z)\,v\\
    &\betac\quad z\,v\quad & &\text{缩减~}(\lambda y.y\,v)\,z\\
    (\lambda x.(\lambda y.y\,x)z)\,v\quad &\betac \quad (\lambda x.z\,x)\,v \quad& &\text{缩减~}(\lambda y.y\,x)\,z\\
    &\betac\quad z\,v\quad & &
  \end{alignat*}
\end{exmp}

从示例 \ref{exmp:difforder} 可以看到,虽然缩减顺序不一样,但最后得
到了同样的结果。这个是不是对所有的 $\lambda$ 项都成立呢?要是不成立,
也就说顺序会影响规约结果,就不好了。这个问题的答案可以说既让人满意也让
人不满意。

\begin{thm}{\bfseries{(Church--Rosser 定理)}}

  若 $P\betar M$ 且 $P\betar N$,则存在一个 $\lambda$ 项 $T$ 使得
$$
M\;\betar\; T\quad\text{且}\quad N\;\betar\; T.
$$
\end{thm}

Church--Rosser 定理最重要的一个应用就是可以用来证明如下重要结果:

\begin{thm}
  若 $P$ 有 $\beta$ 范式,则该范式在模 $\equiv_\alpha$ 的意义下唯一;
  也就是说若 $P$ 有 $\beta$ 范式 $M$ 和 $N$,则 $M\,\equiv_\alpha\,N$。
\end{thm}

有了这个定理,我们可以放一半的心了:如果你对一个 $\lambda$ 项做
$\beta$ 规约,最后得到个范式,可以确定这个范式某种程度上是唯一的,不用
担心你因为规约顺序上的随意性导致这个范式和别人不一样。但为啥说是放一半
的心呢?\emph{因为,这个定理并没有说:若 $P$ 有 $\beta$ 范式,那么你随
  便按任意顺序规约,都能得到 $\beta$ 范式。}

\begin{exmp}\label{exmp:diffdiverge}
  让我们看下面有 $\beta$ 范式的 $\lambda$ 项的例子:
  \begin{alignat*}{2}
    (\lambda x.\lambda y. x)\,a\,((\lambda x.x\,x)\lambda x.x\,x)\; &\betac\;
    ([a/x]\,(\lambda y.x))\,((\lambda x.x\,x)\lambda x.x\,x) &\; &\equiv\; (\lambda y.a)\,((\lambda x.x\,x)\lambda x.x\,x)\\
    &\betac\; [((\lambda x.x\,x)\lambda x.x\,x)/y]\,a & &\equiv\; a
  \end{alignat*}
  这个 $\lambda$ 项其实包含三项,我们要是先归约前两项
 $(\lambda\dots)$,a 就能得到一个 $\beta$ 范式。但注意第三项也是一个
  $\beta$ 可约式,完全可以先对它进行规约。但如我们在示例
  \ref{exmp:beta} 里的倒数第二项看到的,$((\lambda x.x\,x)\lambda
  x.x\,x)$ 会不断规约到自身,所以要是执着于把它先规约了,就永远得不到
  $\beta$ 范式了。
\end{exmp}

于是我们就要问了,若 $P$ 有 $\beta$ 范式,怎么才能规约得到它?按任意顺
序规约就有可能出现示例 \ref{exmp:diffdiverge} 中的问题。可要是全部
穷举,含有 $n$ 个 $\beta$ 可约式的话就有 $n!$ 种规约顺序,一一列举也太
费事了吧。万幸,1958 年 Curry 证明了这么一个定理:

\begin{thm}
  对 $P$ 的总是先 $\beta$ 缩减最左侧最外侧的 $\beta$ 可约式,若这个过
  程能无限进行下去,那么对 $P$ 的所有任意顺序的规约都能无穷进行下去。
\end{thm}

换而言之,要是 $P$ 有 $\beta$ 范式,那么这种最左最外侧优先的规约方式总
能保证得到那个 $\beta$ 范式,这种规约顺序又称为\emph{正则序}。我的解释
器实现的规约顺序就是正则序,虽然它规约需要的步数可能较多,但抵消不了这
个巨大的优势啊。

到目前为止,我们得到的结果都是满意的:一个 $\lambda$ 项若是有 $\beta$
范式,那么这个 $\beta$ 范式某种程度上是唯一的,而且我们有确定的规约顺
序保证得到这个范式。下面就只有一个问题了:能否有一个通用算法,判定一个
$\lambda$ 项是否有 $\beta$ 范式?非常不幸,答案是否定的:

\begin{thm}
  $\lambda$ 项是否有 $\beta$ 范式是不可判定的。
\end{thm}

\begin{rem}

\end{rem}

到这里我们算是粗粗的把 $\lambda$ 演算的主要内容:$\beta$ 规约介绍了一
下。还有个 $\eta$ 规约有兴趣的读者可以自行查找相关资料,这里就不多做介
绍了。
