\section{范畴论基础}

范畴论产生于上世纪40年代对同调代数的研究。经过数十年的发展,范畴论已经成为具有广泛意义的数学理论,在代数学、拓扑学、数理逻辑等领域有深刻的应用。理论计算机科学中,范畴论也广泛用于程序指称语义、程序逻辑、类型理论等领域的研究中。

\subsection{范畴}


\begin{defn}(\textbf{范畴},category)

一个范畴$\mathcal{C}$包括

\begin{tightenum}
    \item Ob$(\mathcal{C})$:一些对象(\textbf{Object})组成的集
    \item Ar$(\mathcal{C})$:对象之间的态射/箭头(\textbf{Morphism}/\textbf{Arrow})组成的集\\
    函数 \textbf{dom} 与 \textbf{cod}:Ar$(\mathcal{C}) \rightarrow$ Ob$(\mathcal{C})$,分别表示一个态射的值域与定义域
    一个从值域 $A$ 到定义域 $B$ 的态射 $f$ 表示为 $f : A \rightarrow B$,值域 $A$ 与定义域 $B$ 之间的态射集表示为 $\mathcal{C}(A,B)$,称为 \textbf{hom-set}
    \item 对任意三个对象 $A, B, C$,态射复合
    \[ \mathcal{C}_{A,B,C} : \mathcal{C}(A,B) \times \mathcal{C}(B,C) \rightarrow \mathcal{C}(A,C)\]
    态射复合 $\mathcal{C}_{A,B,C}(f,g)$ 记为 $g \circ f$ 或 $f\ ;\ g$,图像化表示为 $A \overset{f}\longrightarrow B \overset{g}\longrightarrow C$
    \item 对任意一个对象,存在单位态射 $\text{id}_A : A \rightarrow A$
    \item 对于以上定义,对任意态射 $f : A \rightarrow B$,$g : B \rightarrow C$,$h : C \rightarrow D$,符合公理:
    \[ h \circ (g \circ f) = (h \circ g) \circ f \]
    \[ f \circ id_A = f = id_B \circ f \]
\end{tightenum}

\end{defn}


\subsection{函子}


\begin{defn} (\textbf{函子},functor)
函子可以理解为范畴之间的态射,范畴 $\mathcal{C}$ 与 $\mathcal{D}$ 之间的函子 $F : \mathcal{C} \rightarrow \mathcal{D}$ 定义为:
\begin{itemize}
    \item 对象之间的映射:对于 $\mathcal{C}$ 内的每一个对象 $A$,将其联系至 $\mathcal{D}$ 内的对象 $F\ A$
    \item 态射之间的映射:对于 $\mathcal{C}$ 内的每一个态射 $f : A \rightarrow B$,将其联系至 $\mathcal{D}$ 内的态射 $F\ f : F\ A \rightarrow F\ B$,使得结合律与单位态射得到保留:
    \[ F\ (g \circ f) = F\ g \circ F\ f\]
    \[ F\ id_{A} = id_{F\ A}\]
\end{itemize}
\end{defn}


\subsection{伴随性}

\begin{defn} (\textbf{自然性},naturality)
更进一步地,我们可以定义,自然变化(natural transformation)为函子之间的 ``态射''。若 $F, G : \mathcal{C} \rightarrow \mathcal{D}$ 为范畴 $\mathcal{C}$ 与范畴 $\mathcal{D}$ 之间的函子,$F$ 与 $G$ 之间的自然变化 $t : F \rightarrow G$ 为一个由范畴 $\mathcal{C}$ 中任意对象 $A$ 所 index 的态射族:${\{t_A : F\ A \rightarrow G\ A\}}_{A\in Ob(\mathcal{C})}$。 这个态射族满足:对于范畴 $\mathcal{C}$ 中的所有态射 $f : A \rightarrow B$,使得以下图交换:

\[\xymatrix{
F\ A  \ar[d]^{t_a}  \ar[r]^{F\ f} & F\ B \ar[d]^{t_B}\\
G\ A  \ar[r]^{G\ f} & G\ B
}\]

若对于所有的 $A \in Ob(\mathcal{C})$,$t_A$ 皆为同构,那么 $t$ 就被称为一个自然同构 (natural isomorphism)。
\[ t : F \overset{\cong}\longrightarrow G \]\\

\end{defn}

伴随性可以理解为 Galois Connection 在范畴论之下的抽象。\\

\subsubsection{Galois Connection}

\begin{defn} (\textbf{$g$-逼近})
考虑 $g : Q \rightarrow P$ 为两个偏序集之间的单调映射,对于 $x \in P$,$x$ 的 \textbf{$g$-逼近} 为一个元素 $y \in Q$,使得 $x \le g(y)$。
$x$ 的 \textbf{最佳$g$-逼近} 为 $y \in Q$,使得
\[ x \le g(y) \wedge \forall z \in Q . (x \le g(z) \Rightarrow y \le z) \]
\end{defn}

\begin{defn} (\textbf{Galois Connection})
若存在 $f : P \rightarrow Q$,使得 
\[ \forall x \in P, y \in Q, x \le g(y) \Longleftrightarrow f(x) \le y \]
则 $f$ 为 $g$ 的\textbf{左伴随(left adjoint)},$g$ 为 $f$ 的\textbf{右伴随(right adjoint)}。
\end{defn}

\begin{thm} $g : Q \rightarrow P$ 为两个偏序集之间的单调映射,若存在 $f : P \rightarrow Q$,使得对于任意 $x \in P$ 存在\textbf{最佳$g$-逼近} $f(x)$,则 $f$ 与 $g$ 之间存在 Galois Connection。\\
\end{thm}

证明:
根据定义构造 $f(x) := \min \{\ y \in Q\ |\ x \le g(y)\ \}$

(1) $x \le g(z) \Rightarrow f(x) \le z$.

根据定义,因为 $ x \le g(z)$,所以 $g(z)$ 在集合 $\{\ y \in Q\ |\ x \le g(y)\ \}$ 中,根据构造,$f(x) \le z$。

(2) $f(x) \le z \Rightarrow x \le g(z)$.

根据构造,$x \le g(f(x))$,因为 $g$ 为单调映射以及 $f(x)$ 的构造,$g(f(x)) \le g(z)$。通过偏序的传递性,$x \le g(f(x)) \le g(z)$。$\qed$


\begin{defn} (伴随函子 (adjoint functor))

现在偏序集 $P, Q$ 被抽象为范畴 $\mathcal{C}, \mathcal{D}$,单调函数 $f$ 与 $g$ 抽象为范畴之间的函子:$F : \mathcal{C} \rightarrow \mathcal{D}$,$G : \mathcal{D} \rightarrow \mathcal{C}$,$F$ 与 $G$ 为一对伴随函子若存在双射族 $\theta$,

\[ \theta_{A,B} : \mathcal{C}(A,G(B)) \overset{\cong}\longrightarrow \mathcal{D}(F(A),B)\]
则 $F$ 为 \text{左伴随},$G$ 为 \text{右伴随},记为 $(F \dashv G).f$

\end{defn}

\begin{defn}(笛卡尔闭范畴,CCC)

范畴$\mathcal{C}$是笛卡尔闭范畴(Cartesian closed category, CCC)当且仅当:

\begin{tightenum}
  \item $\mathcal{C}$存在terminal object $T$
  \item 任意两个$\mathcal{C}$的object $X, Y$,有product object $X \times Y \in ob\mathcal{C}$
  \item 任意两个$\mathcal{C}$的object $X, Y$,有exponential object $X^Y \in ob\mathcal{C}$
\end{tightenum}
\end{defn}
