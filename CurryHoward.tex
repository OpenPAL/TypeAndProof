\section{Curry-Howard同构}


下面进入本文的重点:简单类型$\lambda$演算与直觉主义逻辑的Curry-Howard同构。


\subsection{类型和证明}

比较$\lambda$演算的定型规则和上节自然演绎系统的规则,不难得出以下``对应"

\begin{table}[!htb]
\centering
\caption{C-H同构}
\label{my-label}
\begin{tabular}{|l|l|}
\hline
\textbf{类型系统} & \textbf{直觉主义逻辑} \\ \hline
类型            & 公式(命题)            \\ \hline
$\to$类型构造算子        & $\to$逻辑连接词   \\ \hline
变量                        & 假设                  \\ \hline
$\lambda$抽象            & 蕴含引入              \\ \hline
$\lambda$应用            & 蕴含消除              \\ \hline
\end{tabular}
\end{table}


回到本文之前提到的类型居留问题。

\begin{thm}

存在带有特定类型的闭合$\lambda$表达式,当且仅当这个类型对应于一个逻辑定理。
\end{thm}

在直觉主义命题演算中,给定一个导出(或者断言)$x_1:A_1, .., x_n:A_n \vdash B$,存在一个对应的$\lambda$项$M$,且$x_1:A_1,...,x_n:A_n \vdash M : B$。





\subsection{规约和证明}


Refer to book: Lecture on CH Isomorphism?
continuation <-> classical, 
STLC <-> Propositional,
cut elimination <-> reduction
COC + Universe <-> ZFC
System F <-> fragment of Second order intuitionistic logic
Hilbert <-> SKI
UTLC <-> Exfalso (Why total language matter)
