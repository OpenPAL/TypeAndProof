\section{从类型到逻辑}

本文只介绍简单类型$\lambda$演算(STLC)的Curry-Howard同构。在后续发展中,研究人员陆续提出了System F,构造演算等

\subsection{回顾}

我们重新梳理一下$\lambda$演算相关内容,首先是无类型$\lambda$演算。

$$E, F ::= x \ | \ \lambda x.E \ | \ (E F) \ | \ ...$$

然后是简单类型$\lambda$演算。注意,为了表述方便,本节采用和前文略有不同的记法

$$\sigma, \tau ::= \beta \ | \ \sigma \to \tau$$

$$E, F ::= x \ | \ \lambda x : \sigma . E \ | \ E F \ | \ ...$$

定型规则如下:

\begin{prooftree}
\AxiomC{$x : \sigma \in \Gamma$}
\LeftLabel{VAR}
\UnaryInfC{$\Gamma \vdash x : \sigma$}

\end{prooftree}

\begin{prooftree}
\AxiomC{$\Gamma, x : \sigma \vdash E : \tau$}
\LeftLabel{ABS}
\UnaryInfC{$\Gamma \vdash (x : \sigma . E) : (\sigma \to \tau)$}

\end{prooftree}


\begin{prooftree}
\AxiomC{$\Gamma \vdash E : \sigma \to \tau$}
\AxiomC{$\Gamma \vdash F : \tau$}
\LeftLabel{APP}
\BinaryInfC{$\Gamma \vdash E F : \tau$}

\end{prooftree}


\subsection{$\lambda$ Cube概述}




\begin{center}
\xymatrix@!0{
 & \lambda\omega \ar@{-}[rr]\ar@{-}'[d][dd]
  & & \lambda P\omega \ar@{-}[dd]
\\
\lambda2 \ar@{-}[ur]\ar@{-}[rr]\ar@{-}[dd]
  & & \lambda P2 \ar@{-}[ur]\ar@{-}[dd]
\\
 & \lambda\underline\omega \ar@{-}'[r][rr]
  & & \lambda P\underline\omega
\\
 \lambda{\to} \ar@{-}[rr]\ar@{-}[ur]
  & & \lambda P \ar@{-}[ur]
}

\end{center}





$$K ::= \star \ | \ \Box \ | \ K \to K$$

$$T, U ::= V \ | \ S \ | \ T U \ | \ \lambda V : T.U \ | \ \Pi V : T.U$$

\begin{itemize}
  \item Type polymorphism(项依赖于与类型,$\Box \to \star$),如System F,由Girard和Reynolds提出。
  \item Type constructors(类型依赖于类型,$\Box \to \Box$)
  \item Dependent types(类型依赖于项,$\star \to \Box$)
\end{itemize}



\begin{table}[!htb]
\centering
\caption{Eight systems}
\begin{tabular}{|c|c|c|}
\hline
类型系统                            & 关系                                                              & 例                          \\ \hline
$\lambda_{\to}$                 & $\star \to \star$                                               & STLC                       \\ \hline
$\lambda_2$                     & $\star \to \star, \Box \to \star$                               & System F                   \\ \hline
$\lambda _{\underline \omega}$  & $\star \to \star, \Box \to \Box$                                & Weak $\lambda_{\omega}$    \\ \hline
$\lambda_{\omega}$              & $\star \to \star, \Box \to \star, \Box \to \Box$                 & System $\text{F}_{\omega}$ \\ \hline
$\lambda P$                     & $\star \to \star, \star \to \Box$                               & LF                  \\ \hline
$\lambda P_2$                   & $\star \to \star, \star \to \Box, \Box \to \star$               & $\lambda P_2$              \\ \hline
$\lambda P_{\underline \omega}$ & $\star \to \star, \star \to \Box, \Box \to \Box$                 & Weak $\lambda P_{\omega}$  \\ \hline
$\lambda P_{\omega}$            & $\star \to \star, \Box \to \star, \star \to \Box, \Box \to \Box$ & CoC                        \\ \hline
\end{tabular}
\end{table}


\subsection{依赖类型}

\subsection{归纳构造演算和Coq}




\subsection{语义意义下的同构}

\begin{table}[!htb]
\centering
\caption{My caption}
\begin{tabular}{|l|l|l|}
\hline
\textbf{Logic}                                                                & \textbf{Type System}                                                               & \textbf{Category Theory}                                                                     \\ \hline
\begin{tabular}[c]{@{}l@{}}Intuitionistic \\ Propositional Logic\end{tabular} & \begin{tabular}[c]{@{}l@{}}Simple Types\\ Lambda Calculus\end{tabular}             & Cartesian Closed Category                                                                    \\ \hline
\begin{tabular}[c]{@{}l@{}}Second order \\ Intuitionistic Logic\end{tabular}  & System F                                                                           &                                                                                              \\ \hline
Classic Linear Logic                                                          & Linear Type System                                                                 & Star-autonomous Category                                                                     \\ \hline
                                                                              & \begin{tabular}[c]{@{}l@{}}Extensional Dependent \\ Type Theory\end{tabular}       & \begin{tabular}[c]{@{}l@{}}Locally Cartesian \\ Closed Category\end{tabular}                 \\ \hline
                                                                              & \begin{tabular}[c]{@{}l@{}}Homotopy Type Theory \\ without Univalence\end{tabular} & \begin{tabular}[c]{@{}l@{}}Locally Cartesian Closed\\ $(\infty, 1)$-Category\end{tabular} \\ \hline
...                                                                           & ...                                                                                & ...                                                                                          \\ \hline
\end{tabular}
\end{table}

