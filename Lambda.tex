\section{简单类型$\lambda$演算}

本节所采用的带类型$lambda$演算扩展主要参考自\emph{Lectures notes on the Lambda Calculus}, Peter Selinger。

\subsection{语法}

类型的BNF规则如下:

$$A, B ::= \iota \ | \ A \to B \ | \ A \times B  \ | \ 1$$

\begin{tightenum}
  \item 基础类型$\iota$代表integers, booleans等的类型;
  \item $A \to B$为从$A$到$B$的函数的类型;
  \item $A \times B$是二元组$\langle x, y \rangle$ 的类型,其中$x$有类型$A$,$y$有类型$B$。
  \item $1$表示只有一个元素的类型,类似程序语言中的``void"和``unit"。
\end{tightenum}


我们约定:$\times$的结合性比$\to$更强,$\to$是右结合的。比如:$A \times B \to C$等价于$(A \times B) \to C$,$A \to B \to C$相当于$A \to (B \to C)$。


$\lambda$项($\lambda$ terms,或者$\lambda$表达式)的语法:

$$M, N ::= x \ | \ M N \ | \ \lambda x^A . M \ | \ \langle M, M \rangle \ | \ \pi_1 M \ | \ \pi_2 M \ | \ * $$

其中,$\langle M, M \rangle$表示一对$\lambda$项,$\pi_i M$是一个``投射"(projection),定义为:$\pi_i \langle M_1, M_2 \rangle = M_i$。$\star$表示类型``1"的元素(只有一个)。 $\lambda x^A. M$表示$x$有类型$A$。

\begin{rem}

很多文献中对$\lambda x^A. M$采用另一种写法$\lambda x:A. M$
\end{rem}


\begin{defn} (\textbf{自由变量})


\end{defn}

\begin{defn}(\textbf{闭合})

一个$\lambda$项(表达式)是闭合的,如果它不包含自由变量
\end{defn}


\subsection{类型}

从程序语言的角度,类型系统(Type system)属于``静态语义"(static semantic)。下面给出简单类型$\lambda$演算的类型系统

\begin{defn}(\textbf{类型断言},Typing judgement)

又称类型指派,$x:A$:表示变量$x$有类型$A$;
\end{defn}

\begin{defn}(\textbf{类型环境}, Typing environment)

又称类型上下文, 一般用$\Gamma, \Delta,...$ 表示,指代形为$\Gamma = \{x_1:A_1,...,x_k:A_k \}$的类型断言集合,即:$x_1,...x_k$ 分别有类型$A_1,...,A_k$。



\end{defn}

\begin{note}

如果$\Gamma$是一个类型环境,那么$\Gamma, x:A$表示$\Gamma \cup {x:A}$(这样写的时候,总假定$x$不出现在$\Gamma$中)

\end{note}

有了类型环境,类型断言可以表达为,$\Gamma \vdash M:A$,即在类型环境$\Gamma$中, $M$有类型$A$,$\Gamma = \{x_1:A_1,...,x_k:A_k \}$,注意$M$的自由变量必须包含在$x_1, ..., x_k$中。类型环境可以看做是一种假定、前提。


\begin{defn}(\textbf{定型规则},Typing rule)

定型规则指一定类型环境中,表达式类型的推理规则。

\end{defn}

下面是简单类型$\lambda$演算的定型规则:
\begin{tightenum}
  \item $(var)$规则:假定$x$有类型$A$,则$x$有类型$A$。
  \item $(app$规则:类型为$A \to B$的函数可以引用到类型$A$的参数上, 并产生类型$B$的结果
  \item 等等。
\end{tightenum}

$$\frac {}
       {\Gamma, x:A \vdash x : A} \ (var)$$


$$\frac {\Gamma \vdash M : A \to B \ \ \ \ \Gamma \vdash N : A}
       {\Gamma \vdash M N : B} \ (app)$$


$$\frac {\Gamma, x:A \vdash M : B }
       {\Gamma \vdash \lambda x ^ A . M: A \to B} \ (abs)$$

$$\frac {\Gamma \vdash M : A \ \ \ \ \Gamma \vdash N : B }
       {\Gamma \vdash \langle M, N \rangle : A \times B} \ (pair) \ \ \ \ \
\frac {} {\Gamma \vdash \star : 1} \ (\star)$$

$$\frac {\Gamma \vdash M : A \times B }
       {\Gamma \vdash \pi_1 M : A} \ (\pi_1) \ \ \ \ \
\frac {\Gamma \vdash M : A \times B}
       {\Gamma \vdash \pi_2 M : A} \ (\pi_2)$$



\begin{defn}(\textbf{类型导出},Typing derivation)

给定类型断言$\Gamma \vdash M : A$,利用定型规则逐步推出它的过程叫做类型导出。如果存在一个类型导出,
说明该断言是有效的。
\end{defn}

\begin{exmp}
考察下面表达式,我们自底向上对其类型断言给出一个类型导出。
$$\vdash \lambda x^{A \to A} . \lambda y^A . x(xy) : (A \to A) \to A \to A$$


\begin{prooftree}

\AxiomC{}
\UnaryInfC{$x:A \to A, y:A \vdash x:A \to A$}
           \AxiomC{}
           \UnaryInfC{$x:A \to A, y:A \vdash y:A$}
     \BinaryInfC{$x:A \to A, y:A \vdash xy : A$}

     \AxiomC{}
     \UnaryInfC{$x:A \to A, y:A \vdash x:A \to A$}
         \BinaryInfC{$x:A \to A, y:A \vdash x(xy) : A$}
         \UnaryInfC{$x:A \to A, y:A \vdash \lambda y^A . x(xy) : A \to A$}
         \UnaryInfC{$\vdash \lambda x^{A \to A} . \lambda y^A . x(xy) : (A \to A) \to A \to A$}

\end{prooftree}

\end{exmp}


\begin{defn}(\textbf{类型检查},Type checking)

给定$\Gamma$,$M$和$A$,找到一个类型导出。即,如果能够找到,类型检查通过。

\end{defn}


\begin{defn}(\textbf{类型推导},Type inference)

给定$\Gamma$和$M$,确定$A$。

\end{defn}



\begin{defn}(\textbf{良型的},Well-typed)


\end{defn}


\begin{defn}(\textbf{类型可靠性},Type soundness)


\end{defn}

\begin{exmp}
考察以下类型,并思考:是否能够找到闭合的表达式,分别具有以下给定类型?

\begin{tightenum}

  \item $( A \times B) \to A$
  \item $A \to B \to (A \times B)$
  \item $(A \to B) \to (B \to C) \to (A \to C)$
  \item $A \to A \to A$
  \item $((A \to A) \to B) \to B$
  \item $A \to (A \times B)$
  \item $(A \to C) \to C$

\end{tightenum}

我们先给出答案:
\begin{tightenum}
  \item $\lambda x^{A \times B}.\pi_1 x$
  \item $\lambda x^A . \lambda y^B . \langle x, y \rangle$
  \item $\lambda x^{A \to B}. \lambda y^{B \to C} . \lambda z^A . y(xz)$
  \item $\lambda x^A . \lambda y^A . x$ 和 $\lambda x^A . \lambda y^A . y$
  \item $\lambda x^{(A \to A) \to B} . x(\lambda y^A . y)$
  \item 无法找到闭合表达式
  \item 无法找到闭合表达式
\end{tightenum}
\end{exmp}

由上例引出如下问题:

\begin{defn} (\textbf{类型居留问题})

给定类型,是否一定存在相应类型的闭合$\lambda$表达式。找到特定类型的
$\lambda$表达式的问题称为类型居留问题。

\end{defn}



类型居留问题揭示了类型和逻辑之间巧妙的联系。下面将类型表达式中的``$\times$"换成逻辑与符号``$\land$",``$\to$"换成逻辑蕴涵符号`` $\to$",可以获得以下逻辑公式:

\begin{tightenum}
  \item $( A \land B) \to A$
  \item $A \to B \to (A \land B)$
  \item $(A \to B) \to (B \to C) \to (A \to C)$
  \item $A \to A \to A$
  \item $((A \to A) \to B) \to B$
  \item $A \to (A \land B)$
  \item $(A \to C) \to C$
\end{tightenum}

\begin{rem}
有的文献为了方便区分,会用$\supset$或者$\Rightarrow$表示逻辑蕴涵,但是为了符合我们的通常印象且不失统一性,本文用$\to$表示逻辑蕴涵符号。尽管它和类型中的符号$\to$ 重复了,请读者根据上下文区分。
\end{rem}

实际上,1-5都是逻辑有效的(都是永真式,或者重言式),而6和7不是。
后面我们会知道,给定类型,只有当它(通过以上方式)对应的逻辑公式有效时,才能找到相应的$\lambda$表达式!

简单类型$\lambda$演算和逻辑的这种联系(或者对应)称为``Curry-Howard同构"。更准确地说,此处的``逻辑"指代直觉主义命题逻辑。 在后续的章节中我们将进一步阐述。


\subsection{正则性}

有两种正则性,强正则性(Strong Normalization),弱正则性(Weak  Normalization)。
强正则性表示,对于一表达式,无论采用任何规约方式,到最后都会变成值。
弱正则性则是,对于一个表达式,都存在一种规约策略,可以规约成值。

简单类型$\lambda$演算有正则性,证明大体如下:
先定义一个集合R,一个表达式在R里面,当且仅当该表达式是强正则的,并没有自由变量,并且
1:该表达式的类型是基础类型
2:该表达式的类型是函数,并且对于所有输入,如果输入在R,输出也在R

通过在类型推导规则上进行归纳,可以证所有的没有自由变量的表达式都在R里面,所以是强正则的。

对于形式化的证明,可以看 \url {https://www.cis.upenn.edu/~bcpierce/sf/current/Norm.html}